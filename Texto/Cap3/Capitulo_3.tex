\chapter{Filtros Ativos Em Sistemas El�tricos}
\todo[inline]{30 P�gs}



\section{Defini��o de Pot�ncia Ativa, Reativa e Fator de Pot�ncia}

Como forma de entender melhor a qualidade de energia e a opera��o de filtros ativos � necess�rio ter os conceitos de pot�ncia ativa, reativa e fator de pot�ncia . 

\subsection{Pot�ncias em Sistemas Senoidais}

Considerando um circuito monof�sico, senoidal linear e operando em regime permanente, as equa��es da tens�o e corrente s�o expressas por \ref{eq:v(t)} e \ref{eq:i(t)}, respectivamente.

\begin{equation}
v(t) = V_p \cos(\omega t)
\label{eq:v(t)}
\end{equation}

\begin{equation}
i(t) = i_p \cos(\omega t - \phi)
\label{eq:i(t)}
\end{equation}

A pot�ncia instant�nea em um circuito monof�sico � definida segundo a equa��o \ref{eq:pot_inst}.

\begin{equation}
\begin{aligned}
p(t) & = v(t)i(t)\\
	 & = V_p \cos(\omega t) \cdot i_p \cos(\omega t - \phi)\\
	 & = \frac{V_p i_p}{2}[\cos(\phi)+\cos(2 \omega t - \phi)]\\
 	 & = \frac{V_p i_p}{2} \cos(\phi)[1+\cos(2\omega t)] + \frac{V_p i_p}{2} \sin(\phi)\sin(2 \omega t)
\end{aligned}
\label{eq:pot_inst}
\end{equation}

A equa��o \ref{eq:pot_inst} pode ser dividido em dois termos variantes no tempo: o primeiro � dado por 

\begin{equation}
\dfrac{V_p i_p}{2} \cos(\phi)[1+\cos(2\omega t)]
\label{eq:prim}
\end{equation}

e o segundo por.
\begin{equation}
\dfrac{V_p i_p}{2} \cdot \sin(\phi)\sin(2 \omega t)
\label{eq:sec}
\end{equation}

Por defini��o, a pot�ncia ativa � definida pelo valor m�dio da equa��o \ref{eq:prim}, e a pot�ncia reativa � definida pelo valor de pico da equa��o \ref{eq:sec}.
\todo{verificar sobre valores de pico e valores eficazes}




\subsection{Defini��o de Pot�ncias em Sistemas N�o-Senoidais}

\subsection{Pot�ncia Instant�nea Utilizando a Teoria P-Q}

A teoria p-q � baseada na transforma��o das tens�es e correntes das coordenadas $abc$ para $\alpha \beta 0$
\subsubsection{Transformada de Clarke}

\begin{equation}
\begin{bmatrix}
v_0\\
v_\alpha\\
v_\beta
\end{bmatrix}
= \sqrt{\dfrac{2}{3}}
\begin{bmatrix}
\dfrac{1}{\sqrt{2}}	& \dfrac{1}{\sqrt{2}}	& \dfrac{1}{\sqrt{2}}		\\[2ex]
1					& -\dfrac{1}{2}			& -\dfrac{1}{2}				\\[2ex]
0					& \dfrac{\sqrt{3}}{2}	& -\dfrac{\sqrt{3}}{2}
\end{bmatrix}
\begin{bmatrix}
v_a\\
v_b\\
v_c
\end{bmatrix}
\end{equation}

\section{Filtros Ativos}

\subsection{Filtros Ativo Empregando a Teoria P-Q}

%de acordo com os paranaues, \cite{Watanabe1991}, \cite{Akagi2007}, \cite{Watanabe2004}, \cite{Afonso2000}, \cite{Couto2003}

