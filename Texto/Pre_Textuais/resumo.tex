


Aqui come�a o resumo do referido trabalho. N�o tenho a menor id�ia do que colocar aqui. Sendo assim, vou inventar. L� vai: Este trabalho apresenta uma metodologia de controle de posi��o das juntas passivas de um manipulador subatuado de uma maneira sub�tima. O termo subatuado se refere ao fato de que nem todas as juntas ou graus de liberdade do sistema s�o equipados com atuadores, o que ocorre na pr�tica devido a falhas ou como resultado de projeto. As juntas passivas de manipuladores desse tipo s�o indiretamente controladas pelo movimento das juntas ativas usando as caracter�sticas de acoplamento da din�mica de manipuladores. A utiliza��o de redund�ncia de atua��o das juntas ativas permite a minimiza��o de alguns crit�rios, como consumo de energia, por exemplo.
Apesar da estrutura cinem�tica de manipuladores subatuados ser id�ntica a do totalmente atuado, em geral suas carater�sticas din�micas diferem devido a presen�a de juntas passivas. Assim, apresentamos a modelagem din�mica de um manipulador subatuado e o conceito de �ndice de acoplamento. Este �ndice � utilizado na sequ�ncia de controle �timo do \mbox{manipulador}.
A hip�tese de que o n�mero de juntas ativas seja maior que o n�mero de
passivas  $(n_{a} > n_{p})$  permite o controle �timo das juntas passivas, uma vez que na etapa
de controle destas h� mais entradas (torques nos atuadores das juntas ativas), que
elementos a controlar (posi��o das juntas passivas). 