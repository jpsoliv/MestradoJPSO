The evolution in aeronautical systems has brought an increase in the use of electrical equipment and has become a trend in the aviation market, where there is an inclination to fulfill projects with enphasis in the More Electric Aircraft (MEA) concept. However, The increasing of the electrical system dependence, along with the increase of the amount of electrical load connected in the distribution system, has raised the concern to the issues related to power quality, which is associated with the increase of the harmonic distorition in the voltage waveforms. To ensure the propoer power quality, this subject is covered by aeronautical standards e shall be followed in the development of aircraft systems, in order to ensure its operational safety. In this context, this study lists the main solutions used to mitigate the harmonic components and brings a comparison with their respective characteristics, emphasizing the benefits and deficiencies for each solution. In this scenario, this study focuses in the power quality improvement and power factor correction by the utilization of active filtering. For this understanding, this study presents the instantaneous power theory, as well as the main theoretical basis used in its comprehension and elaboration. To validade its implementation in an aeronautical electrical system, a simulation is proposed with the active filter operating in a power generation and distribution system, where the filter is connected in the power input of an electro hydrostatic actuator. The models used in the simulation intent to simulate, with sufficient levels of details, an aeronatucal electrical system, and the results obtained are presented as a mean to measure the effectiveness of the active filter implementation.