The new aircraft developments have undergone an evolution in the definition of its embedded systems, which there is an tendency in the increase of electrical equipment use. This has become a common sense in new developments and is a trend in the aviation market, where there is an inclination to fulfill projects with emphasis in the More Electric Aircraft (MEA) concept. However, the increasing of the electrical system dependence, along with the expansion of the amount of electrical load connected in the distribution system, has raised the concern to the issues related to power quality, which is associated with the growth of the harmonic distortion in the voltage waveforms. To ensure the proper power quality, this subject is covered by aeronautical standards and shall be followed in the development of aircraft systems, in order to ensure its operational safety. In this context, this study lists the main solutions used to mitigate the harmonic components and brings a comparison with their respective characteristics, emphasizing their benefits and deficiencies for each solution. In this scenario, this study focuses in the power quality improvement and power factor correction by the utilization of active filtering. The understanding of the instantaneous power theory, as well as the main theoretical basis are used in comprehension and elaboration of the filters. To validate its implementation in the aeronautical electrical system, a simulation is proposed with the active filter operating in a power generation and distribution system, where the filter is connected at the power input of an electro hydrostatic actuator. The models used in the simulation intent to simulate, with sufficient levels of details, an aeronautical electrical system, and the results obtained are presented as a mean to measure the effectiveness of the active filter implementation.