The evolution in aeronautical systems has brought an increasing in the use of electrical equipment and has become a trend in the aviation market, where there is an inclination to fulfill projects with enphasis in the More Electric Aircraft (MEA) concept. However, The increasing of the electrical system dependence, along with the increase of the amount of electrical load connected in the distribution system, has raised the concern to the issues related to the power quality, which is associated with the increase of the harmonic distorition in the voltage waveforms.
%\textit{The evolution of aeronautical systems has brought an increase in the use of electrical equipment and has become a trend in the aviation market, where there is a propensity to carry out projects with emphasis on the concept of \textit {More Electric Aircraft} (MEA). However, the increase in the dependence of the electrical system, together with the increase in the number of loads connected in the network, has raised the concern with the problems related to the quality of energy coming from the insertion of harmonic components in the voltage waveforms.} 
The condition of ensuring the quality of energy is required by aeronautical standards and should be considered in the development of an aircraft in order to ensure its operational safety. In this context, this work lists the main solutions to mitigate the presence of harmonic components and brings a comparison with their main characteristics, emphasizing the strengths and weaknesses of each solution. In this scenario, the focus of the study is directed to the improvement of power quality and correction of the power factor with the use of active filtering. For the understanding of the development of active filters, the theory of instantaneous power is presented, as well as the main theoretical bases to be discussed in their comprehension and elaboration. As a way of validating its implementation in aeronautical systems, a simulation is proposed with the operation of a generation and distribution system operating with active type tit shunt filters on the input of electrohydrostatic actuators. The models used in the elaboration of the simulation intend to simulate adequately the operation of an electric system, and the results obtained are presented and used as an instrument in the discussion of the effectiveness of the implementation of the active filter.