Primeiramente, gostaria de agradecer aos meu pais, \\
pelo apoio incondicional dado n�o apenas na longa jornada da p�s-gradua��o, mas tamb�m em todas as escolhas e decis�es que tomei at� ent�o. 

Ao Prof. Dr. Roberto D'amore, \\ 
pela orienta��o e confian�a depositada no desenvolvimento desse trabalho, al�m de proferir conselhos e sugest�es �mpares em seu desenvolvimento.

Ao Eng. MSc. Andr� Domingues Rocha de Oliveira, \\ 
por aceitar o desafio em co-orientar o desenvolvimento dessa disserta��o, emitindo sugest�es valiosas e conhecimento em sua elabora��o.

Aos membros do time de Engenharia El�trica da Embraer Defesa e Seguran�a, \\
pelos ensinamentos imprescind�veis passados ao longo do per�odo de p�s-gradua��o, em que resultou no amadurecimento do tema aqui proposto. 

E n�o menos importante, ao povo brasileiro\\
que paga os devidos impostos e sustenta uma institui��o de ensino superior como o ITA, com um dos mais altos padr�es de ensino superior do Brasil.
