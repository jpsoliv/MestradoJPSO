\chapter{Qualidade de Energia em Aeronaves}

O mercado da avia��o tem passado por uma mudan�a nos preceitos de desenvolvimento de sistemas que v�o desde a utiliza��o de novas tecnologias embarcadas at� a mudan�a na concep��o de opera��o da aeronave. Essa tend�ncia vem ocorrendo de maneira natural como evolu��o do mercado pela demanda de aeronaves mais eficientes e competitivas. Nesse contexto h� o conceito de \textit{More Electric Aircraft} (MEA). Como o pr�prio nome diz, essa concep��o baseia-se em aeronaves cuja filosofia de projeto contempla o uso abundante de sistemas alimentados eletricamente com o objetivo de aumentar a efici�ncia. Isso pode ser visto nos mais recentes desenvolvimentos de aeronaves, como por exemplo o Boeing 787, onde a redu��o da emiss�o de CO\textsubscript{2} � 20\% menor se comparado com o Boeing 767. \cite{Boeing2007}. O ganho n�o se d� apenas na redu��o da emiss�o de gases pela queima de combust�veis f�sseis, mas tamb�m pro bla bla bla \cite{Boeing2007}.
