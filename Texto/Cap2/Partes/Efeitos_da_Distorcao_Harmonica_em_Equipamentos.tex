\section{Efeitos da Distor��o Harm�nica em Equipamentos}

Equipamentos el�tricos que constituem um sistema qualquer necessitam de alimenta��o el�trica para funcionar. Entretanto, para que o equipamento entregue as fun��es desejadas e tenha seu funcionamento adequado, as tens�es no ponto de entrada de energia devem ser conforme as especifica��es requeridas do fabricante do equipamento. Para o caso em estudo, onde a tens�o � alternada com frequ�ncia constante, a alimenta��o deve entregar n�veis de tens�o e frequ�ncia bem estabelecidos para atender a certos crit�rios de qualidade, de modo a n�o danificar os sistemas conectados � rede.

Com a inser��o de cargas n�o lineares na rede, surgem distor��es na forma de onda da tens�o que abate na qualidade de energia da aeronave. Para o caso de aeronaves, aplicando a s�rie de Fourier na ondula��o da tens�o, idealmente espera-se que haja apenas uma componente senoidal em 400 Hz, por�m, devido as cargas n�o lineares conectados � rede, h� o aparecimento de componentes em frequ�ncias m�ltiplas de 400 Hz, denominadas harm�nicas. A presen�a de harm�nicas na rede do sistema el�trico acaba por distorcer a forma de onda senoidal tornando-a disforme e alterando seus n�veis de tens�o.



\cite{Wagner1993}
