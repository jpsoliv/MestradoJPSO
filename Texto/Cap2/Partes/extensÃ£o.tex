Encontram-se inúmeras topologias de conversores com correção de fator de potência na literatura. O mais utilizado é o conversor com correção de fator de potência, ou \textit{Poewer Factor Coreection} (PFC) do tipo \textit{boost}. Esse conversor é concebido pela ponte de diodos arranjados juntamente com chaves controladamente comutadas. Ainda, pela flexibilidade de possíveis arranjos dos semicondutores sob a ponte de diodos e pelas linhas de entrada e saída de energia, uma gama imensa de conversores do tipo \textit{PFC boost} pode ser concebido. Para elucidar o funcionamento deste tipo de conversor será estudado o conversor Prasad-Ziogas, cuja topologia é simples e apresenta uma única chave controlada. Também para facilitar o entendimento, será explicado a operação em modo de condução descontínua. O circuito deste conversor é mostrado na figura \ref{fig:}. O princípio de funcionamento é dado pelo controle da chave $S_1$ que, quando em estado de condução, aplica a tensão da fonte sobre os indutores de entrada $L_i,\;i=1,2,3$. Isso faz com que a corrente nos indutores cresçam de forma proporcional a tensão aplicada em seus terminais pela fonte de entrada. Quando o a chave $S_1$ para de conduzir, a corrente dos indutores são levadas à zero. Por ter a freqüência de comutação muito maior que a freqüência da rede, tem-se que a corrente de entrada apresenta uma forma modulada em alta freqüência e facilmente filtrada com filtros passa-baixa \cite{Nairus1996, Takeuchi2008}. Com isso é possível que a corrente apresente formato senoidal e em fase com a tensão. Na prática o funcionamento desse conversor apresenta certa distorção harmônica e não suporta alta carga de potência, de modo que outras topologias hibridas com circuitos \textit{PFC boost} e \textit{PFC buck} sejam aplicadas, ou ainda a conexão em delta ou estrela das fases em separado para aumentar a capacidade de potência e tornar o FP próximo da unidade \ref{Kola2011}.
\todo{separar as figuras dos conversores e simular o conversor para colocar as forams de onda da bagaça}

@techreport{Nairus1996,
  title={Three-Phase Boost Active Power Factor Correction for Diode Rectifiers.},
  author={Nairus, John G},
  year={1996},
  institution={AFRL Propulsion Directorate - Wright-Patterson Air Force Base},
  address={Ohio}
}
