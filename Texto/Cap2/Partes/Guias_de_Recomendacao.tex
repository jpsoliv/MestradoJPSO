\section{Guias de Recomendação}

Aeronaves civis necessitam de certificado de tipo para ter autorização de voar no espaço aéreo. Ainda, com o intuito de aumentar a integração de aeronaves civis com militares, além de melhorar os níveis de segurança operacional destas, há uma tendência de exigência de certificação civil para aeronaves militares pelas forças aéreas \ref{\todo{Paulo Serra corp.}}. Além do mais, estas aeronaves possuem exigências de cumprimento de normas e requisitos específicos para aumentar os níveis de segurança no cumprimento das possíveis missões designadas a elas.

Com o objetivo de auxiliar no desenvolvimento e padronizar procedimentos de projeto, processos de manufatura, manutenção, e testes de qualificação, há certos documentos que são usualmente utilizados com o intuito de guiar o desenvolvimento e especificar requisitos de segurança e operação. Contudo, esses documentos não são necessariamente requisitos de certificação para qualificar as aeronaves perante os órgãos reguladores. Todavia, o cumprimento dessas normas é de papel fundamental para facilitação na aquisição do certificado de tipo da aeronave.

Na indústria aeronáutica, existem vários órgão que instituem e publicam documentos com recomendações e padronizações de modo a auxiliar o desenvolvimento de aeronaves mais seguras. Alguns órgãos reguladores publicam alguns desses documentos, como o FAA (\textit{Federal Aviation Administration}) que tem publicado o AC’s (\textit{Advisory Circular}). Ainda, existe o departamento de defesa dos Estados Unidos (DoD - \textit{Department of Defence}) com as MIL-STD’s (\textit{Military Standard}). Outros órgãos não relacionados com agentes governamentais que instituem recomendações de práticas aeronáuticas são o SAE (\textit{Society of Automotive Engineering}) e o RTCA (\textit{ Radio Technical Commission for Aeronautics}), onde este ultimo possui as DO’s. Tais documentos, diferentemente das FAR’s (\textit{Federal Aviation Regulation}), não são requisitos diretos para obtenção de certificado de tipo, porém a realização destas trazem padronizações que facilitam a obtenção da certificação de tipo das aeronaves.

Com relação ao projeto e desenvolvimento de sistemas elétricos de aeronaves, existem vários documentos que buscam a definição e padronização dos elementos e das características esperadas constituintes do sistema elétrico. Com relação à qualidade de energia da aeronave, os dois principais documentos são a MIL-STD 704 e a DO-160 (\textit{Section 16}) que definem limites e parâmetros que asseguram o bom funcionamento do sistema elétrico.

\subsection{MIL-STD 704 - \textit{Aircraft Electric Power Characteristics}}

\subsection{DO-160 - \textit{Environmental Conditions and Test Procedures for Airborne Equipment}}
