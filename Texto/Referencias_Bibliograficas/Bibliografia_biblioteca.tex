\providecommand{\abntreprintinfo}[1]{%
 \citeonline{#1}}
\begin{thebibliography}{}
%	\raggedright						% Não JUSTIFICADO, SEGUNDO A BIBLIOTECA NÂO JUSTIFICA ESSA PARTE
%	\itemindent0cm
	\providecommand{\abntrefinfo}[3]{}
	\providecommand{\abntbstabout}[1]{}
	\abntbstabout{1.42.2.1 }

\bibitem{Babikian2002} %[1]
\abntrefinfo{Babikian, Lukachko e Waitz}{BABIKIAN; LUKACHKO; WAITZ}{2002}
{BABIKIAN, R.; LUKACHKO, S.~P.; WAITZ, I.~A. The historical fuel efficiency
  characteristics of regional aircraft from technological, operational, and
  cost perspectives.
\textbf{Journal of Air Transport Management}, v.~8, n.~6, p. 389--400,
  2002.}

\bibitem{Moir1999}		%[2]
\abntrefinfo{Moir}{MOIR}{1999}
{MOIR, I. More-electric aircraft-system considerations. In:  \uppercase{IEEE
  Colloquium on Electrical Machines and Systems for the More Electric
  Aircraft}, 1999. \textbf{Proceedings...} Londres: IET, 1999.}

\bibitem{Abdelhafez2009}	%[3]
\abntrefinfo{Abdel-Hafez e Forsyth}{ABDEL-HAFEZ; FORSYTH}{2009}
{ABDEL-HAFEZ, A.; FORSYTH, A. A review of more-electric aircraft. In:
  \uppercase{International Conference on Aerospace Science \& Aviation
  Technology, 13\textsuperscript{\lowercase{th}}}, 2009. \textbf{Proceedings...} Cairo: Military Technical College, 2009.}

\bibitem{Abdel2012}			%[4] %%REVER ISSO
\abntrefinfo{Abdel-Hafez}{ABDEL-HAFEZ}{2012}
{ABDEL-HAFEZ, A. Power generation and distribution system for a
	more electric aircraft: a review. In: AGARWAL, R. \textbf{Recent Advances in Aircraft Technology}.
	Ar\'abia Saudita: INTECH, 2012. Chap. 13}

\bibitem{Boeing2007}		%[5]
\abntrefinfo{Karimi}{KARIMI}{2007}
{KARIMI, K.~J. \textbf{Future aircraft power systems:} integration challenges. [S.l.]: The Boeing Company, 2007.}

\bibitem{Akagi1984}			%[6]
\abntrefinfo{Akagi, Kanazawa e Nabae}{AKAGI; KANAZAWA; NABAE}{1984}
{AKAGI, H.; KANAZAWA, Y.; NABAE, A. Instantaneous reactive power 
  compensators comprising switching devices without energy storage components.
  \textbf{IEEE Transactions on industry applications}, IEEE, n.~3, p. 625--630,
  1984.}

\bibitem{Akagi2007}			%[7]
\abntrefinfo{Akagi, Watanabe e Aredes}{AKAGI; WATANABE; AREDES}{2007}
{AKAGI, H.; WATANABE, E.~H.; AREDES, M. \textbf{Instantaneous power theory and
  applications to power conditioning}. Chichester: John Wiley \& Sons, 2007.}

\bibitem{Frost2008}			%[8]
\abntrefinfo{Srimoolanathan}{SRIMOOLANATHAN}{2008}
{SRIMOOLANATHAN, B. \textbf{Aircraft electrical power systems:} charged with
  opportunities. 2008.
  Dispon{\'\i}vel em:
  $<$https://www.frost.com/sublib/display-market-insight.do?id=150507057$>$.
  Acessado em 29/03/2015.}

\bibitem{Avery2007}			%[9]
\abntrefinfo{Avery, Burrow e Mellor}{AVERY; BURROW; MELLOR}{2007}
{AVERY, C.; BURROW, S.; MELLOR, P. Electrical generation and distribution for
  the more electric aircraft. In:  \uppercase{International
  Universities Power Engineering Conference, 42\textsuperscript{\lowercase{nd}}}, 2007. \textbf{Proceedings...} Piscataway: IEEE, 2007. p.
  1007--1012.}

\bibitem{Zhao2014}			%[10]
\abntrefinfo{Zhao, Guerrero e Wu}{ZHAO; GUERRERO; WU}{2014}
{ZHAO, X.; GUERRERO, J.~M.; WU, X. Review of aircraft electric power systems
  and architectures. In: \uppercase{IEEE International Energy Conference}, 2014, Dubrovnik. \textbf{Proceedings...}
  Piscataway: IEEE, 2014. p. 949--953.}

\bibitem{Singer2012}		%[11]
\abntrefinfo{Singer et al.}{SINGER et al.}{2012}
{SINGER, C. et al. Aircraft electrical power systems and nonlinear dynamic
  loads.
\textbf{SAE International Journal of Aerospace}, v.~5, n.~2,
  p. 447--454, 2012.}

\bibitem{Linear}			%[12]
\abntrefinfo{Equipamentos El\'etricos e Eletr\^onicos de Pot\^encia
  Ltda}{EQUIPAMENTOS EL\'ETRICOS E ELETR\^ONICOS DE POT\^ENCIA LTDA}{}
{EQUIPAMENTOS EL\'ETRICOS E ELETR\^ONICOS DE POT\^ENCIA.
  \textbf{Harm\^onicos em instala\c{c}\~oes el\'etricas}.
  Dispon{\'\i}vel em:
  $<$http://www.engematec.com.br/site/downloads/ harmonicos\_em\_instalacoes\_eletricas.pdf$>$.
  \newblock Acessado em 06/06/2016.}

\bibitem{Cidade}			%[13]
\abntrefinfo{Cidade}{CIDADE}{}
{CIDADE, G. \textbf{Eletricidade e eletr\^onica aplicada \`a bioci\^encias}.
  Rio de Janeiro: UFRJ, 2006. Apostila.
  Dispon{\'\i}vel em:
  $<$http://fisbio.biof.ufrj.br/restrito/bmb353/4\_M\_eletric/ele\_ele/conc\_bas/ index.htm$>$.
  \newblock Acessado em 25/04/2015.}

\bibitem{Alexander2005}		%[14]
\abntrefinfo{Alexander e Sadiku}{ALEXANDER; SADIKU}{2005}
{ALEXANDER, C.~K.; SADIKU, M. N.~O. \textbf{Fundamentals of electric
  circuits}. 3\textsuperscript{rd} ed. [S.l.]: McGraw-Hill, 2005.}

\bibitem{Wagner1993}		%[15]
\abntrefinfo{Wagner et al.}{WAGNER et al.}{1993}
{WAGNER, V. et al. Effects of harmonics on equipment.
\textbf{IEEE Transactions on Power Delivery}, v.~8, n.~2, p. 672--680,
  1993.}

\bibitem{Pomilio2010}		%[16]
\abntrefinfo{Deckmann e Pomilio}{DECKMANN; POMILIO}{2010}
{DECKMANN, S.~M.; POMILIO, J.~A. \textbf{Avalia{\c{c}}{\~a}o da qualidade da
  energia el{\'e}trica}. 2010.
  Dispon{\'\i}vel em:
  $<$http://www.dsce.fee.unicamp.br/antenor/pdffiles/qualidade/b5.pdf$>$.
  Acessado em 28/05/2015.}


\bibitem{Kassick2010}		%[17]
\abntrefinfo{Kassick}{KASSICK}{2010}
{KASSICK, E.~V. \textbf{Harm{\^o}nicas em sistemas industriais de baixa
  tens{\~a}o}.
Florian{\'o}polis: Instituto de Eletr\^onica de Pot\^encia,
  Universidade Federal de Santa Catarina.}

\bibitem{Pomilio2014}		%[18]
\abntrefinfo{Pomilio}{POMILIO}{2014}
{POMILIO, J.~A. \textbf{Conversores com outras t{\'e}cnicas de
  comuta{\c{c}}{\~a}o suave}. 2014.
  Dispon{\'\i}vel em:
  $<$http://www.dsce.fee.unicamp.br/~antenor/pdffiles/CAP5.pdf$>$.}
  Acessado em 28/05/2015.


\bibitem{Manousaka2013}		%[19]
\abntrefinfo{Manousaka}{MANOUSAKA}{2013}
{MANOUSAKA, E.
\textbf{DC-DC buck converter with inrush current limiter}. 2013.
Thesis (Master in Science in Sustanible Energy Technology) -- Faculty of Applied Sciences, TUDelfet, Lorentzweg.}

\bibitem{Notches}			%[20]
\abntrefinfo{Rockwell Automation}{ROCKWELL AUTOMATION}{}
{ROCKWELL AUTOMATION. \newblock \textbf{Eliminating voltage notching on the distributions
  system}.
  Dispon{\'\i}vel em:
  $<$http://www.ab.com/support/abdrives/documentation/techpapers/ notch.htm$>$.}
  Acessado em 29/05/2015.


\bibitem{Fitzgerald2006}	%[21]
\abntrefinfo{Fitzgerald, Kingsley e Umans}{FITZGERALD; KINGSLEY; UMANS}{2006}
{FITZGERALD, A.; KINGSLEY, C.; UMANS, S. \textbf{M{\'a}quinas el{\'e}tricas:} com
  introdu{\c{c}}{\~a}o {\`a} eletr{\^o}nica de pot{\^e}ncia. 6\textsuperscript{th} ed. Porto Alegre:
  Bookman, 2006.}

\bibitem{Ary2011}			%[22]
\abntrefinfo{Victorino}{VICTORINO}{2011}
{VICTORINO, A. \textbf{Fator de pot\^encia e distor\c{c}\~ao harm\^onica}. 2011.
	Dispon{\'\i}vel em:
	$<$http://www.joinville.ifsc.edu.br/~aryvictorino/leituras\_SIP\_2011-1$>$. 
	Acessado em 18/06/2016.}


\bibitem{AN779}				%[23]
\abntrefinfo{Lacanette}{LACANETTE}{1991}
{LACANETTE, K. \textbf{A Basic introduction to filters:} active, passive, and
  switched-capacitor. National Semiconductor 1991. (Application Note 779).}

\bibitem{Mussoi2004}		%[24]
\abntrefinfo{Mussoi e Esperan{\c{c}}a}{MUSSOI; ESPERAN{\c{C}}A}{2004}
{MUSSOI, F.~L.; ESPERAN{\c{C}}A, C. \textbf{Resposta em frequ{\^e}ncia:} filtros
  passivos.
2\textsuperscript{nd} ed. Florian{\'o}polis: Centro Federal de Educa{\c{c}}{\~a}o
  Tecnol{\'o}gica de Santa Catarina, 2004.}

\bibitem{Singh2008}			%[25]
\abntrefinfo{Singh et al.}{SINGH et al.}{2008}
{SINGH, B. et al. Multipulse {AC-DC} converters for improving power quality: a
  review.
\textbf{IEEE Transactions on Power Electronics}, v.~23, n.~1, p. 260--281,
  2008.}

\bibitem{Gong2003}			%[26]
\abntrefinfo{Gong, Drofenik e Kolar}{GONG; DROFENIK; KOLAR}{2003}
{GONG, G.; DROFENIK, U.; KOLAR, J. 12-pulse rectifier for more electric
  aircraft applications. In: \uppercase{IEEE International Conference on
  Industrial Technology}, 2003, Maribor. \textbf{Proceedings...} 
  Piscataway: IEEE, 2003. v.~2, p. 1096--1101.}

\bibitem{Lobo2005}			%[27]
\abntrefinfo{Gong et al.}{GONG et al.}{2005}
{GONG, G. et al. Comparative evaluation of three-phase high-power-factor
  {AC-DC} coverter concepts for application in future more electric aircraft.
\textbf{IEEE Transactions on Industrial Electronics}, v.~52, n.~3, p.
  727--737, 2005.}

\bibitem{Kolar2011}			%[28]
\abntrefinfo{Kolar e Friedli}{KOLAR; FRIEDLI}{2011}
{KOLAR, J.~W.; FRIEDLI, T. The essence of three-phase {PFC} rectifier systems.
  In: \uppercase{International Telecommunications Energy
  Conference}, 33\textsuperscript{\lowercase{rd}}., 2011, Amsterdam. \textbf{Proceedings...} 
  Piscataway: IEEE, 2011. p. 1--27.}

\bibitem{Barbosa2002}		%[29]
\abntrefinfo{Barbosa}{BARBOSA}{2002}
{BARBOSA, P.~M.
\textbf{Three-phase power factor correction circuits for low-cost distributed
  power systems}. 2002. Thesis (PhD in Electrical Engineering) -- Faculty of the Virginia Polytechnic Institute, Blacksburg.}

\bibitem{Nairus1996}		%[30]
\abntrefinfo{Nairus}{NAIRUS}{1996}
{NAIRUS, J.~G. \textbf{Three-phase boost active power factor correction for diode
  rectifiers.} Wright-Patterson Air Force Base,
  Ohio: AFRL Propulsion Directorate, 1996.}

\bibitem{Takeuchi2008}		%[31]
\abntrefinfo{Takeuchi et al.}{TAKEUCHI et al.}{2008}
{TAKEUCHI, N. et al. A novel PFC circuit for three-phase utilizing a single
  switching device. In: \uppercase{International Telecommunications
  Energy Conference, 30\textsuperscript{\lowercase{th}}.,} 2008, San Diego. \textbf{Proceedings...} 
  Piscataway: IEEE, 2008. p.~1--5.}

\bibitem{Pomilio2009}		%[32]
\abntrefinfo{Pomilio e Deckmann}{POMILIO; DECKMANN}{2009}
{POMILIO, J.~A.; DECKMANN, S.~M. \textbf{Condicionamento de energia el{\'e}trica
  e dispositivos FACTS}. Campinas: LCEE-DSCE-FEEC-UNICAMP, 2009. (IT-741)}

\bibitem{Afonso2013}		%[33]
\abntrefinfo{Afonso, Gon{\c{c}}alves e Pinto}{AFONSO; GON{\c{C}}ALVES;
  PINTO}{2013}
{AFONSO, J.~L.; GON{\c{C}}ALVES, H.; PINTO, J. Active power conditioners to
	 mitigate power quality problems in industrial facilities.
	 In: ZOBAA, A. \textbf{Power quality issues}.
     [S.l.]: Intech Open Access Publisher, 2013. Chap. 5.}

\bibitem{Zhu2014}			%[34]
\abntrefinfo{Zhu e Ma}{ZHU; MA}{2014}
{ZHU, S.; MA, W. Methods of aircraft grid harmonic reduction: A review.
\textbf{Scholars Journal of Engineering and Technology}, v.~2, p.
  270--275, 2014.}

\bibitem{Barruel2004}		%[35]
\abntrefinfo{Barruel, Schanen e Retiere}{BARRUEL; SCHANEN; RETIERE}{2004}
{BARRUEL, F.; SCHANEN, J.; RETIERE, N. Volumetric optimization of passive
  filter for power electronics input stage in the more electrical aircraft. In:
  \uppercase{Annual Power Electronics Specialists Conference, 35\textsuperscript{\lowercase{th}}., 2004.}
  \textbf{Proceedings...} Piscataway: IEEE, 2004. v.~1, p. 433--438.}

\bibitem{Chen2012research}	%[36]
\abntrefinfo{Chen et al.}{CHEN et al.}{2012}
{CHEN, Z. et al. A research on cascade five-level aeronautical active power
  filter. In: \uppercase{International Power Electronics and Motion
  Control Conference,} 7\textsuperscript{th}., 2012. \textbf{Proceedings...} Piscataway: IEEE, 2012. v.~4, p. 2732--2737.}

\bibitem{Akagi2006}			%[37]
\abntrefinfo{Akagi}{AKAGI}{2006}
{AKAGI, H. Modern active filters and traditional passive filters.
\textbf{Bulletin of the Polish Academy of Sciences}, v.~54,
  n.~3, 2006. (Technical Sciences)}

\bibitem{Chen2012novel}		%[38]
\abntrefinfo{Chen e Chen}{CHEN; CHEN}{2012}
{CHEN, Z.; CHEN, M. A novel 400Hz shunt active power filter for aircraft
  electrical power system. In: \uppercase{International Power Electronics and Motion
  	Control Conference,} 7\textsuperscript{th}., 2012. \textbf{Proceedings...} Piscataway: IEEE, 2012. v.~4, p.
  2838--2843.}

\bibitem{Chen2012control}	%[39]
\abntrefinfo{Chen, Luo e Chen}{CHEN; LUO; CHEN}{2012}
{CHEN, Z.; LUO, Y.; CHEN, M. Control and performance of a cascaded shunt active
  power filter for aircraft electric power system.
\textbf{IEEE Transactions on Industrial electronics}, v.~59, n.~9, p.
  3614--3623, 2012.}

\bibitem{Karatzaferis2013}	%[40]
\abntrefinfo{Karatzaferis et al.}{KARATZAFERIS et al.}{2013}
{KARATZAFERIS, J. et al. Comparison and evaluation of power factor correction
  topologies for industrial applications.
\textbf{Energy and Power Engineering}, v.~5, n.~6, p. 401--410, 2013.}

\bibitem{Paredes2012}		%[41]
\abntrefinfo{Paredes}{PAREDES}{}
{PAREDES, H. K.~M. \textbf{Eletr\^onica de pot\^encia para gera\c{c}\~ao,
  transmiss\~ao e distribui\c{c}\~ao de energia el\'etrica}: t\'opicos em
  teorias de pot\^encia em condi\c{c}\~oes n\~ao ideais de opera\c{c}\~ao. 2012.
  Dispon{\'\i}vel em:
  $<$http://www.dsce.fee.unicamp.br/~antenor/pdffiles/it744/CAP6.pdf$>$.}
  Acessado em 24/06/2016.

\bibitem{Staudt2008}		%[42]
\abntrefinfo{Staudt}{STAUDT}{2008}
{STAUDT, V. Fryze-Buchholz-Depenbrock: a time-domain power theory. In:
	\uppercase{International School on Nonsinusoidal Currents and Compensation}, 2008.
	\textbf{Proceedings...} Piscataway: IEEE, 2008. p. 1--12.}

\bibitem{Czarnecki1997}		%[43]
\abntrefinfo{Czarnecki}{CZARNECKI}{1997}
{CZARNECKI, L. Budeanu and Fryze: two frameworks for interpreting power
  properties of circuits with nonsinusoidal voltages and currents.
\textbf{Electrical Engineering}, v.~80, n.~6, p. 359--367, 1997.}

\bibitem{Budeanu1927}		%[44]
\abntrefinfo{Budeanu}{BUDEANU}{1927}
{BUDEANU, C. \textbf{Puissances r{\'e}actives et fictives}.
	Bucharest. Romania: Institut Romain De I'energte, 1927.}

\bibitem{Czarnecki1987}		%[45]
\abntrefinfo{Czarnecki}{CZARNECKI}{1987}
{CZARNECKI, L.~S. What is wrong with the Budeanu concept of reactive and
  distortion power and why it should be abandoned.
\textbf{IEEE Transactions on Instrumentation and measurement}, v.~1001,
  n.~3, p. 834--837, 1987.}

\bibitem{AredesSD}			%[46]
\abntrefinfo{Watanabe e Aredes}{WATANABE; AREDES}{}
{WATANABE, E.~H.; AREDES, M. \textbf{Teoria de pot\^encia ativa e reativa
  instant\^anea e aplica{\c{c}}{\~o}es:} filtros ativos e FACTS. 1998. 
  Curso de curta dura{\c{c}}{\~a}o ministrado/Extens{\~a}o. Dispon\'ivel em:
  $<$http://ifgjatai.webcindario.com/watanabe2.pdf$>$. 
  Acessado em 04/05/2016.}

\bibitem{Peng1996}			%[47]
\abntrefinfo{Peng e Lai}{PENG; LAI}{1996}
{PENG, F.~Z.; LAI, J.-S. Generalized instantaneous reactive power theory for
  three-phase power systems.
\textbf{IEEE Transactions on Instrumentation and Measurement}, v.~45, n.~1,
  p. 293--297, 1996.}

\bibitem{Afonso2000}		%[48]
\abntrefinfo{Afonso, Couto e Martins}{AFONSO; COUTO; MARTINS}{2000}
{AFONSO, J.~L.; COUTO, C.; MARTINS, J.~S. Active filters with control based on
  the pq theory.
\textbf{IEEE Industrial Electronics Society Newsletter}, v.~47, n.~3, p.
  5--10, 2000.}

\bibitem{Haberberger2004}	%[49]
\abntrefinfo{Haberberger e Fuchs}{HABERBERGER; FUCHS}{2004}
{HABERBERGER, M.; FUCHS, F.~W. Novel protection strategy for current
  interruptions in IGBT current source inverters. In: \uppercase{Annual 
  Power Electronics Specialists Conference,} 35\textsuperscript{th}., 2004 
  \textbf{Proceedings...} Piscataway: IEEE, 2004. v.~1, p. 558--564.}

\bibitem{Trzynadlowski2015}	%[50]
\abntrefinfo{Trzynadlowski}{TRZYNADLOWSKI}{2015}
{TRZYNADLOWSKI, A.~M. \textbf{Introduction to modern power electronics}.
	3\textsuperscript{rd} ed. New York: John Wiley \& Sons, 2015.}

\bibitem{Thekkevalappil2005}	%[51]
\abntrefinfo{Thekkevalappil}{THEKKEVALAPPIL}{2005}
{THEKKEVALAPPIL, S.~N.
Master Thesis, \textbf{Hysteretic pulse width modulation with internally
  generated carrier for a boost DC-DC converter}. 2005. Thesis (Master in Science) --
  University of Florida, Gainesville.}

\bibitem{Watanabe1993}	%[52]
\abntrefinfo{Watanabe, Stephan e Aredes}{WATANABE; STEPHAN; AREDES}{1993}
{WATANABE, E.~H.; STEPHAN, R.~M.; AREDES, M. New concepts of instantaneous
  active and reactive power in electrical systems with generic loads.
\textbf{IEEE Transactions on Power Delivery}, v.~8, n.~2, p. 697--703,
  1993.}

\bibitem{Olivier}		%[53]
\abntrefinfo{Tremblay}{TREMBLAY}{}
{TREMBLAY, O; DESSAINT, L. \textbf{Aircraft Electrical Power Generation and
  Distribution}. 2016.
  Dispon{\'\i}vel em:
$<$https://www.mathworks.com/help/physmod/sps/examples/aircraft-electrical- power-generation-and-distribution.html$>$.}
  Acessado em 14/10/2016.


\bibitem{IEEE}			%[54]
\abntrefinfo{IEEE\ldots}{IEEE\ldots}{1992}
{INSTITUTE OF ELECTRICAL AND ELECTRONICS ENGINEER. \textbf{IEEE Std 421.5-1992}:
  recommended practice for excitation system models for power system
  Stability Studies. New York, 1992.}

\bibitem{Exner1953}		%[55]
\abntrefinfo{Exner e Singer}{EXNER; SINGER}{1953}
{EXNER, D.; SINGER, G. Impedance data for 400-cycle aircraft distribution
  systems.
\textbf{Transactions of the American Institute of Electrical Engineers, Part II:}
  Applications and Industry, v.~71, n.~6, p. 410--419, 1953.}

\bibitem{Dinca2014}		%[56]
\abntrefinfo{Dinca et al.}{DINCA et al.}{2014}
{DINCA, L. et al. Mathematical modeling and analysis of an electro-hydrostatic
  servo actuator with brushless {DC} motor. In: \uppercase{Modern
  Computer Applications in Science and Education}, 2014. \textbf{Proceedings...} Cambridge, MA: WSEAS
  Press, 2014. p. 157--163.}

\bibitem{MILSTD}		%[57]
\abntrefinfo{Department of Defence Interface Standard}{DEPARTMENT OF DEFENCE
  INTERFACE STANDARD}{2004}
{UNITED STATES. Department of Defence Interface Standard. \textbf{MIL-STD 704F:2004}: aircraft
  electric power characteristics. Washington, DC, 2004.}

\end{thebibliography}
