\section{Introduction}

The increase of the aircraft operational costs associated with the fuel consumption makes this subject one of the main concern in the development of the new aircraft projects \citep{Babikian2002}. In this scenario, the aviation market has been changed the design perception with respect of use of the electrical system. The electrical system dependency to power an increasing number of embedded systems and, in some cases, replacing the power source where it used to be powered by hydraulic and pneumatic system has increased in the past few years, creating the concept of the More Electrical Aircraft (MEA) \citep{Moir1999}.

This context raised the relevance of the electrical system in the role of aircraft operational safety. In this way, the electrical system needs to have a greater reliability and to operate in a way to avoid failures of the equipment connected to it. However, the rise of electrical equipment connected in the electrical system, specially the non-linear loads, has increased the harmonic distortion content being introduced in the electrical grid \citep{Singer2012}, diminishing the power quality and becoming a subject of study in aircraft operational safety.

To improve the power quality with the reduction of the total harmonic distortion (THD), some conditioners are applied in the equipment power input and in the electrical grid. The implementation of these conditions must considers the reliability, the weight and cost to be feasible in aircraft systems.

In this context, some topologies to increase the power quality are already use in the aircraft electrical system, such as the passive filters and multi-pulse converters \citep{Zhu2014,Gong2003,Lobo2005}. However, its characteristics are unfavorable to extensive use due to its weight and volume, making these applicable only to specific equipment.

With the increase in the number of the non-linear loads applied in the electrical grid, and the requirement to ensure the good power quality, this paper presents a concept of a shunt active filter to be applied in the electrical aircraft system. This subject is being an item of recent study considering different active filters topologies \citep{Chen2012research,Chen2012novel,Chen2012control}. To understand the theory evolved in the active filter operation, this paper also presents the instantaneous power theory, as well as its physical implementation. A simulation is presented to analyze the shunt active filter operation when considered its use in an aircraft electrical power system.
