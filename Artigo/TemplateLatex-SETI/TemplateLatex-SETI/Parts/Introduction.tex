\section{Introduction}

The increase of the aircraft operational costs associated with the fuel consumption drives the development of new aircraft technologies \citep{Babikian2002}. In this scenario, the aviation market has changed the design perception regarding the electrical system, replacing hydraulic and pneumatic’s power source to equivalent electrical ones, creating the concept of the More Electrical Aircraft (MEA) \citep{Moir1999}.

This context raised the relevance of the EPGDS (Electrical Power Generation and Distribution System) in the role of aircraft operational safety. This way, the electrical system needs to have a greater reliability and operates to avoid failures of the equipment connected to the grid. However, the increase of the non-linear loads connected to the electrical system has raised the harmonic distortion content introduced in the EPGDS \citep{Singer2012}, diminishing the power quality and becoming a subject of study in aircraft operational safety. 

To improve the power quality with the reduction of the total harmonic distortion (THD), some conditioners must be applied in the equipment power input and in the electrical grid. The implementation of these conditioners must considers the reliability, weight and cost to be feasible in aircraft systems.

In this context, some topologies to increase the power quality are already used in the aircraft electrical system, such as the multi-pulse rectifiers \citep{Zhu2014,Gong2003,Lobo2005}. However, its weight and volume make this topology applicable only to specific equipment.

With the increase of the non-linear loads applied to the electrical grid, along with the requirements to ensure power quality, some alternatives have been proposed. In this scenario, the use of a shunt active filter applied in an aircraft electrical system is an item of recent study, considering different active filters topologies \citep{Chen2012research,Chen2012novel,Chen2012control}.

This article analyses the use of a shunt active filter to improve the power quality in the aircraft EPGDS. It starts by reviewing the active filter operation, the instantaneous power theory, and continues discussing control techniques. A simulation is presented to analyze the shunt active filter operation with three electrohydraulic actuators (EHA), which are non-linear loads used to control the aircraft latero-directional and longitudinal aerodynamics surfaces. The simulation model is compounded by the electrical generation and distribution system connected to EHAs with its respective shunt active filters.