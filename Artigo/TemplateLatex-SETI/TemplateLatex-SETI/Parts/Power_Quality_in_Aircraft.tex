\section{Power Quality in Aircraft}\label{sec:Power_Quality}

The power quality in aircraft EPGDS is a concern which regards the airworthiness. The electrical equipment embedded in aircraft must be qualified to ensure the proper operation and integration. Thereby, the power quality is one of the subjects considered in the qualification tests, which are specified by standard test procedures issued by aeronautical authorities. The most used qualifications standards for electrical systems are the MIL-STD 704, which qualifies the EPGDS, and the DO-160 - Section 16, which qualifies the embedded electrical equipment. To ensure the proper equipment integration, the EPGDS and equipment must comply with these standards.

The non-linear loads inject harmonic distortion content in the system, inducing degradation in the power quality and decreasing the power factor. Furthermore, the increase in the number of electrical equipment connected in the grid enhances the degradation of the power quality. Thus, techniques for power factor correction must be applied in the EPGDS to limit the system parameters within the constraints of aeronautical standards.

Some techniques are already used in aircraft electrical system to increase the power factor and power quality. One of these techniques is the use of multi-pulse converters, which is most employed in high current rectifiers to improve the power quality. However, despite of the good reliability, they are bulky and heavy. There are some other topologies that are useful for harmonic content reduction, but their characteristics do not make them feasible to operate in the aircraft systems. Some of these topologies are the passive filters and power factor correction (PFC) converters. For the passive filters, despite of good reliability and low cost, the high weight is the main problem to its implementation in aircraft \citep{Barruel2004}. For the PFC converters, the downside lies in the low reliability and low density of energy conditioned \citep{Zhu2014,Gong2003,Lobo2005}. 

In this scenario, the active filter, due to its features as lightweight and fast response to load variation, appears to be a feasible topology to reduce the harmonic content and increase the power factor \citep{Zhu2014,Chen2012control,Karatzaferis2013}. There are some drawbacks in its use, as the high complexity and low reliability. However, the advances in power electronics are making them practical to be implemented in aircraft electrical system \citep{Abdelhafez2009}.
