\section{Power Qaulity in Aircraft}

The power quality in aircraft electrical generation and distribution system is a concern which regards its safe airworthiness. The electrical fed equipment used in aircraft design must be qualified to ensure the proper operation and integration. Thereby, the power quality is one of the subjects considered in the qualification tests, which are specified by standard test procedures issued by aeronautical authorities. The most used standards in qualification are the MIL-STD 704, which qualifies the electrical system; and the DO-160, which qualifies the embedded equipment. Considering an electrical power generation and distribution system (EPGDS) operating without complying with these standards, the equipment connected to the grid are not ensured to operate properly. Thus, means of power factor correction must be applied in the EPGDS in order to adequate the system to operates within the constraints of the aeronautical standards.

Some implements are already used in aircraft electrical system to reduce the power factor and increase the power quality. Some of these artifacts are the passive filters and multi-pulse converters \cite{passive} \cite{Gong2003,Lobo2005}, however, despite of the good reliability, they are bulky and heavy. There are some other topologies that are useful for harmonic content reduction, but, their characteristics do not make them feasible to operate in the aircraft systems. In this scenario, the active filter, due to its features as lightweight and fast response to load variation, appears to be a feasible topology to reduce the harmonic content and increase de power factor \cite{Zhu2014,Chen2012control,Karatzaferis2013}. There are some drawbacks in its use, as the high complexity and low reliability, but the advances in power electronics are making them feasible to be implemented in aircraft electrical system \cite{Abdelhafez2009}.
