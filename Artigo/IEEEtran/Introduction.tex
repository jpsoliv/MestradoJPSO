\section{Introduction}

The increase of the aircraft operational costs associated with the fuel consumption makes this subject one of the main concern in the development of the new aircraft projects. In this scenario, the aviation market has been changed the design perception with respect of use of the electrical system. The electrical system dependency to power an increasing number of embedded systems and, in some cases, replacing the power source where it used to be powered by hydraulic and pneumatic system has increased in the past few years, creating the concept of the More Electrical Aircraft (MEA).

This context raised the relevance of the electrical system in the hole of aircraft operational safety. In this way, the electrical system needs to have a greater reliability and to operate in a way to avoid failures of the equipment connected to it. However, the rise of electrical equipment connected in the electrical system, specially the non-linear loads, has increased the harmonic distortion content being introduced in the electrical grid, diminishing the power quality and becoming a subject of study in aircraft operational safety. To allow the systems equipment and electrical systems proper integration, some documents were released to standardize the electrical parameters, such as the DO-160 and the MIL-STD 704. These documents bring the electrical parameter acceptance, and one of these constraints are regarding of the power quality and Total Harmonic Distortion (THD). \todo[inline]{talvez seja retirado}

To improve the power quality with the decreasing of the THD, some power conditioners are applied in the equipment power input and in the electrical power distribution systems, such as filters and high power factor converters. However, the implementation of these conditioners has a drawback, which is the increase of the weight, volume and complexity, whereas the reliability decreases.

\todo[inline]{colocar as devidas referências}
\todo[inline]{terminar esse capítulo de acordo com o conteúdo que será apresentado no restante do artigo}

A introdução deve conter três pontos importantes. 
Motivação do trabalho: interesse, aplicações possíveis e problema sendo resolvido. 
Realizações anteriores: mencione artigos que descrevam trabalhos semelhantes (mesmo problema ou outras soluções). Não descreva com detalhes as realizações anteriores, destaque pontos importantes que deixem claro a contribuição do seu trabalho. 
Contribuição do trabalho: qual a novidade que está sendo proposta (uma nova solução, uma nova arquitetura, um desempenho melhor etc.).



%REFERENCES
%
%Procure referências de artigos (base de dados da CAPES), livros etc. Referências da Internet podem ser empregadas com cuidado. 
%
%Todas referências citadas devem estar no texto.
